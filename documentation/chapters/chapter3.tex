\chapter{System Analysis}

\section{System Requirements}
A requirement is simply a statement of what the system must do or what characteristics
it needs to have. During a systems development project, requirements will be created
that describe what the business needs (business requirements), what the users
need to do (user requirements), what the software should do (functional requirements),
characteristics the system should have (nonfunctional requirements), and how
the system should be built (system requirements).

\subsection{Functional Requirements}
Functional requirements are the specifications of the product’s functions
(features). In another words, functional requirements define what precisely a
software must do and how the system must respond to inputs. Functional
requirements define the software's goals, meaning that the software will not work if
these requirements are not met.
\begin{enumerate}
    \customitem{Authentication}
        \begin{enumerate}[label*=\arabic*.]
            \item The system will allow users to create an account.
            \item The system must validate users credentials to login.
            \item Users should have the ability to reset their passwords in case of forgotten credentials. 
        \end{enumerate}
    \customitem{Enroll Courses}
        \begin{enumerate}[label*=\arabic*.]
            \item Students can enroll courses and see course content
              We will extend this to enroll paid course wit credit card, debit card etc. in the future.
        \end{enumerate}
    \item \textbf{Course Managment} \hfill
        \begin{enumerate}[label*=\arabic*.]
            \item Instructors can create a course and upload to our website.
            \item Instructors can upload various types of course content, including text, multimedia, and documents.
            \item Only authorized instructors should have the ability to modify or update course content.
        \end{enumerate}
\end{enumerate}

\subsection{Non-Functional Requirements}
Non-functional Requirements define system attributes such as security,
reliability, performance, maintainability, scalability, and usability. They serve as
constraints or restrictions on the design of the system across the different backlogs.
Also known as system qualities, non-functional requirements are just as critical as
functional requirements. They ensure the usability and effectiveness of the entire
system. They specify criteria that judge the operation of a system, rather than specific
behaviors.

\begin{itemize}
    \customitem{Performance} \vspace{0.2cm} \\
        The application should be able to handle large numbers of concurrent users.  
    \customitem{Security} \vspace{0.2cm} \\
        The application should protect user data from
        unauthorized access or theft.
    \customitem{Scalability} \vspace{0.2cm} \\
        The application should be able to handle an increasing number of users and services.
    \customitem{User-friendliness} \vspace{0.2cm} \\
        The application should be easy to use and navigate, with
        clear instructions and explanations of the analysis process.
    \customitem{Privacy} \vspace{0.2cm} \\
        The application should have a clear privacy policy and
        should not retain user data
    
\end{itemize}

\section{Process of Requirements }
